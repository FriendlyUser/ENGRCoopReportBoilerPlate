%%%% Useful Packages loaded without much configuration %%%%%%
\usepackage[margin=2.54 cm]{geometry} % Set dimensions for page layout
\usepackage{kpfonts}	    % Fonts used in the title page
\usepackage{eso-pic}        % Background pictures in the title page
\usepackage{tcolorbox}		% Fancy Math equations	
\usepackage{tabularx}		% Tabulars with adjustable-width columns
\tcbuselibrary{skins} 	% used with tcolorbox 
\usepackage{transparent}	% Create transparent background images
\usepackage{lipsum}  % Garbage text
\usepackage{url}	 % links to websites
\usepackage{pdfpages} % to include pdf pages
\usepackage{booktabs}	% Make high quality tables
\usepackage{colortbl}	% Color tables
\usepackage{xfrac}		% Make fractions nice in tables
\usepackage{enumitem}	% itemize, enumerate
\usepackage{array}
\usepackage{setspace}   % double spacing as required in UVIC co-op reports
\usepackage{titlesec} % Select alternative section titles
\titlespacing*{\section}
{0pt}{5.5ex plus 1ex minus .2ex}{4.3ex plus .2ex}	% Not sure about this spacing
\usepackage{multirow} % Create tabular cells spanning multiple rows.

%CHANGE Link colours as needed
\definecolor{linkColor}{HTML}{453737}	

\usepackage{hyperref}	% Create links in document
\hypersetup{linktocpage}	% Allow clickable links
\hypersetup{
    bookmarks=true,         % show bookmarks bar?
    unicode=false,          % non-Latin characters in Acrobat’s bookmarks
    pdftoolbar=true,        % show Acrobat’s toolbar?
    pdfmenubar=true,        % show Acrobat’s menu?
    pdffitwindow=false,     % window fit to page when opened
    pdfstartview={FitH},    % fits the width of the page to the window
    pdftitle={Work Term Report},    % title
    pdfauthor={David Li},     % author	
    %linktoc=all,     %set to all if you want both sections and subsections linked
	colorlinks   = true, %Colours links instead of ugly boxes
	urlcolor     = linkColor, %Colour for external hyperlinks
	linkcolor    = linkColor, %Colour of internal links
	citecolor   = linkColor %Colour of citations
}


\usepackage{float}    % used for float­ing ob­jects such as fig­ures and ta­bles.
\usepackage[figurename=Figure]{caption}
\captionsetup[figure]{labelfont=sf,textfont={sf}}
\captionsetup[table]{labelfont=sf,textfont={sf}}

\usepackage{parskip} % no indents
\usepackage{tikz}	 % for drawings and cool graphics

%% Math packages %%
\usepackage{siunitx}	% SI units
\usepackage{mathtools}	% LOAD MATH
\usepackage{amssymb}  	% MATH Symbols


%%%%%%%%%% COLORS %%%%%%%%%%%%%%%%%%%%%%%%%%%%%
\definecolor{titlepagecolor}{cmyk}{1,.60,0,.40}

%%%%%%%%%%%%%%%%%%%%%%%%%%%%%%%%%%%%%%%%%%%%%%%%%%%%%%%%%%%%%%%%%%%%%%
%%%%%%%%%%%%%%%%%%%%%%%%%%%% CODE STYLINGS %%%%%%%%%%%%%%%%%
%%%%%%%%%%%%%%%%%%%%%%%%%%%%%%%%%%%%%%%%%%%%%%%%%%%%%%%%%%%%%%%%%%%%%%%%

\usepackage{listings} % For source code
\DeclareCaptionFont{white}{\color{white}}
\DeclareCaptionFormat{listing}{\colorbox{titlepagecolor}{\parbox{1\textwidth}{#1#2 \quad #3}}}
\captionsetup[lstlisting]{format=listing,labelfont=white,textfont={white,sf}} % for fancy boxes

\lstdefinestyle{default}{frame=tb,
	aboveskip=3mm,
	backgroundcolor=\color{cornsilk},
	belowskip=3mm,
	showstringspaces=false,
	columns=flexible,
	basicstyle={\ttfamily},
	numbers=none,
	numberstyle=\tiny\color{red},
	keywordstyle=\color{blue},
	commentstyle=\color{green},
	stringstyle=\color{purple},
	morekeywords={fclose, exit, printf, fscanf, strcpy, strlen,
		strcmp, fprintf},
	breaklines=true,
	breakatwhitespace=true,
	tabsize=3,
	captionpos=t,
	columns=flexible,
}
%%%%%%%%%%%%%%%%%%%%%%%%%%%%%%%
%%%%%% SQL style setup %%%%%%%%
\makeatletter
\newcommand{\lstuppercase}{\uppercase\expandafter{\expandafter\lst@token
		\expandafter{\the\lst@token}}}	%%%UPPERCASE SQL%%%%
\newcommand{\lstlowercase}{\lowercase\expandafter{\expandafter\lst@token
		\expandafter{\the\lst@token}}} %%% REGULAR CASE %%%%
\makeatother

\definecolor{mauve}{rgb}{0.58,0,0.82}
\lstdefinestyle{Oracle}{basicstyle=\ttfamily,
	keywordstyle=\lstuppercase,
	backgroundcolor=\color{cornsilk},
	emphstyle=\itshape,
	showstringspaces=false,
	morekeywords={ACCESS, MOD, NLS_DATE_FORMAT, NVL, REPLACE, SYSDATE,
		TO_CHAR, TO_NUMBER, TRUNC},
	numberstyle=\tiny\color{black},
	keywordstyle=\color{red},
	commentstyle=\color{green},
	stringstyle=\color{mauve},
	columns=flexible,
}
%%%%%%%%%%%%%%%%%%%%%%%%%%%%%%%%%%%%%%%%%%%%%%%%
%%%%%%%%%%%%%%%%%%%%%JAVA %%%%%%%%%%%%%%%%%%%%%%
\definecolor{javared}{rgb}{0.6,0,0} % for strings
\definecolor{javagreen}{rgb}{0.25,0.5,0.35} % comments
\definecolor{javapurple}{rgb}{0.5,0,0.35} % keywords
\definecolor{javadocblue}{rgb}{0.25,0.35,0.75} % javadoc
\lstdefinestyle{myJava}{frame=tb,
	basicstyle=\ttfamily,
	backgroundcolor=\color{cornsilk},
	keywordstyle=\color{javapurple}\bfseries,
	stringstyle=\color{javared},
	commentstyle=\color{javagreen},
	morecomment=[s][\color{javadocblue}]{/**}{*/},	% Make comments blue
	morecomment=[is]{/*}{*/},  % Remove comments
	stepnumber=2,
	numbers=left,    % print line numbers
	numbersep=10pt,
	tabsize=4,
	showspaces=false,
	showstringspaces=false,
	linewidth=\textwidth,
	columns=flexible,			% Important for keeping text in the frame
	breaklines=true,
}

%%%%%%%%%%%%%%%%%% ENDJAVA %%%%%%%%%%%%%%%%%%%%%
%%%%%%%%%%%%%%%%%%%%%%%%%%%%%%%%%%%%%%%%%%%%%%%%

%%%%%%%%%%%%%%%%%%%%%%%%%%%%%%%%%%%%%%%%%%%%%%%%%%
%%%%%%%%%%%%%%%%%%%%% JAVASCRIPT%%%%%%%%%%%%%%%%%%%
\lstdefinelanguage{JavaScript}{frame=tb,
	keywords={typeof, new, true, false, catch, function, return, null, catch, switch, var, if, in, while, do, else, case, break},
	keywordstyle=\color{blue}\bfseries,
	ndkeywords={class, export, boolean, throw, implements, import, this},
	ndkeywordstyle=\color{darkgray}\bfseries,
	identifierstyle=\color{black},
	sensitive=false,
	comment=[l]{//},
	morecomment=[s]{/*}{*/},
	commentstyle=\color{purple}\ttfamily,
	stringstyle=\color{red}\ttfamily,
	morestring=[b]',
	morestring=[b]"
}
\definecolor{cornsilk}{rgb}{1.0, 0.97, 0.86}
\lstdefinestyle{myJavaScript}{
	language=JavaScript,
	backgroundcolor=\color{cornsilk},
	extendedchars=true,
	basicstyle=\ttfamily,
	showstringspaces=false,
	showspaces=false,
	tabsize=2,
	breaklines=true,
	showtabs=false,
	captionpos=t,
}
%%%%%%%%%%%%%%%%%%%%%%%%%%%%%%%%%%%%%%%%%%%%%%%%%%
%%%%%%%%%%%%%% END CODE STYLINGS %%%%%%%%%%%%%%%%%
%%%%%%%%%%%%%%%%%%%%%%%%%%%%%%%%%%%%%%%%%%%%%%%%%

%%%%%%%%%%%%%%%%%%%%%%%%%%%%%%%%%%%%%%%%%%%%%%%%%%%%%%%%%%%%
%%%%%%%%%%%%%%%%%%%%%%%%% FORMATTING %%%%%%%%%%%%%%%%%%%%%%%
% Format TOC, header, footers and chapters
%% tocloft settings
\usepackage[titles]{tocloft}	% Pro­vides con­trol over the ty­pog­ra­phy of the Ta­ble of Con­tents, List of Fig­ures and List of Tables
\setlength{\cftbeforesecskip}{3pt}
% End tocloft settings

\usepackage{sectsty} % used to color chapter and sections
%\renewcommand{\cftpartleader}{\cftdotfill{\cftdotsep}} % for parts
%\renewcommand{\cftchapleader}{\cftdotfill{\cftdotsep}} % for chapters
\renewcommand{\cftsecleader}{\cftdotfill{\cftdotsep}} % for sections, if you really want! (It is default in report and book class (So you may not need it).
% ----------------------------------------------------------------
%\renewcommand{\cftdotsep}{0}	 % dots in the title page
% \renewcommand{\cftsectleader}{\bfseries\cftdotfill{\cftsecdotsep}}% dot leaders in bold 


   % NUMBERING IN TOC FOR SUMMARY SECTION %%%%%%%%
   \newcommand{\mysection}[2]{
   	\setcounter{section}{#1}
   	\setcounter{subsection}{0}
   	\section*{#2}
   	\addcontentsline{toc}{section}{#2}
   }
  
   %%%%%%%%%%%%%%%%%%%%%%%%%%%%%%%%
    %CHAPTER Title formatting
	\definecolor{myText}{HTML}{2B2B2B}
	\definecolor{myChap}{HTML}{000066}
	\definecolor{mySect}{HTML}{336699}
	\definecolor{mySubSect}{HTML}{2088B2}

    \renewcommand{\sectionmark}[1]{\markboth{\thesection.\ #1}{}}
    
    % SET UP COLOURS FOR HEADINGS IN THE REPORT
    \sectionfont{\color{mySect}}  % sets colour of sections
    \subsectionfont{\color{mySubSect}}  % sets colour of subsections
    \subsubsectionfont{\color{mySubSect}}  % sets colour of subsubsections
    \subparagraphfont{\color{red}}		    % sets colour of paragraphs

   	%%%%%%%%%%%%%%%%%%%%%%%%%%%%%%%%%%%%%%%%%%%%%%%%%%%%%%%%%%%%%%%%
   	%CHANGE  GLOBAL TEXT COLOR %%%%%%%%%%%%%%%%%%%%%
   	\makeatletter
   	\newcommand{\globalcolor}[1]{%
   		\color{#1}\global\let\default@color\current@color
   	}
   	\makeatother
   	
   	% Edit the global text colour, make the report easier to read for digital screens or print.
   	\AtBeginDocument{\globalcolor{myText}}
   	
   %%%%%%%%%%%%%%%%%%%%%%%%%%%%%%%%%%%%%%%%%%%%%%%%%%%%%%%
   % SET UP COMBINED LOT and LOF 	%%%%%%%
   %	CONFIG LOL					%%%%%%%
   %%%%%%%%%%%%%%%%%%%%%%%%%%%%%%%%%%%%%%%%%%%%%%%%%%%%%%%%%%%
%   \makeatletter
%   \def\ext@figure{lot}
%   \makeatother\

%	\setcounter{tocdepth}{4} % INCLUDE SUBSUBSECTIONS IN TOC
  % \renewcommand{\lstlistingname}{Listing}	% Change caption in code listing
   %\renewcommand{\lstlistlistingname}{List of Scripts}     % CHANGE THE HEADER IN TOC
   
   % Create List of Figures and Tables in LATEX
%   \renewcommand*\listtablename{List of Figures and Tables}
%   \renewcommand{\cftfigpresnum}{Figure~}
%   \renewcommand{\cftfigaftersnum}{:}
%   \setlength{\cftfignumwidth}{5.5em}
%   \renewcommand{\cfttabpresnum}{Table~}		
%   \renewcommand{\cfttabaftersnum}{:}
%   \setlength{\cfttabnumwidth}{5.5em}
   
%   %CONFIG SCRIPTS in TOC %%%%%%
%   \makeatletter
%   \AtBeginDocument{%	Add the word Program in front of the caption in TOC and listing
%   	\renewcommand\lstlistoflistings{\bgroup
%   		\let\contentsname\lstlistlistingname
%	\def\l@lstlisting##1##2{\@dottedtocline{1}{1.3em}{3em}{\bfseries Script 
%				##1}{##2}}
%   		\let\lst@temp\@starttoc \def\@starttoc##1{\lst@temp{lol}}%
%   		\tableofcontents \egroup}
%   }
%   \makeatother
%   
%   %MAKE SPACING THE SAME AS IN THE LOT AND LOF in Programs %%%%%%
%%   \makeatletter
%%   \let\my@chapter\@chapter
%%   \renewcommand*{\@chapter}{%
%%   	\addtocontents{lol}{\protect\addvspace{10pt}}%
%%   	\my@chapter}
%%   \makeatother
   
%%%%%%%%%%%%%%%%%%%%%%%%%%%%%%%%%%%%%%%%%%%%%%%%%%%%%%%%%%%%%%%%%%
%%%%%% SET UP BIBLIOGRAPHY  %%%%%%%%%%%%%%%%%%%%%%%%%%%%%%%%%%%%%%%%%%%
%%%%%%%%%%%%%%%%%%%%%%%%%%%%%%%%%%%%%%%%%%%%%%%%%%%%%%%%%%%%%%%%%%
\usepackage[backend=bibtex,sorting=none]{biblatex}	% Sort by citation order
\addbibresource{bibliography.bib}		% Load bibliography
\usepackage[nottoc,numbib]{tocbibind}	% Label the Biblography in the TOC
\usepackage[titletoc]{appendix}	%% appendix will be in the toc
\defbibheading{bibliography}[\bibname]{%
  	\section{#1}%	% set up bibliography as a section 
  	\markboth{\thesection.\ #1}{}}	%label at upper right corner in header
\newcommand{\onlineCite}{[Online] Available: }	% Used in BIBLIOGRAPHY 


%%%%%%%%%%%%%%%%%%%%%%%%%%%%%%%%%%%%%%%%%%%%%%%%%%%%%%%%%%%%%%%%%%%
%%%%%%%%% CUSTOMIZE GLOSSARY %%%%%%%%%%%%%%%%%%%%%%%%
\usepackage[toc,nopostdot,xindy]{glossaries} % included in toc and keep track of technical words and definition and remove dots at the end

% glossary definitions as bold
\defglsentryfmt{\color{black}\bfseries\glsgenentryfmt}

% custom glossary style that contains the pages that contains the glossary terms, and bolded glossary entries.
\newglossarystyle{myGloss}{%
	\setglossarystyle{list}
	\renewcommand*{\glossentry}[2]{%
		\item[\textbf{\glsentryitem{##1}}%
		\textbf{\glstarget{##1}}{\textbf{\glossentryname{##1}:}}]
		\glossentrydesc{##1}\glspostdescription\space See p.\space ##2} 
}
%\setglossarystyle{listhypergroup}
\setglossarystyle{myGloss}

% Set up roman numercals page numbering for front matter, letter of transmittal, and table of contents.
\newcommand\frontmatter{%
	\cleardoublepage
	%\@mainmatterfalse
	\pagenumbering{roman}}
\renewcommand{\familydefault}{\sfdefault} % Nice font formatting

\let\cleardoublepage\clearpage % prevent book from create extra pages bewteen chapters

%%% Set up verbatim for WYSIWYG formatting for letter
\usepackage[T1]{fontenc}%  selects EC fonts
\usepackage{verbatim}%     configurable verbatim
\makeatletter
\def\verbatim@font{\normalfont% select the font
	\let\do\do@noligs
	\verbatim@nolig@list}
\makeatother

%%%%%%%%%%%%%%%%%%%%%%%%%%%%%%%%%%%%%%%%%%%%%%%%%%
%%%%%%%%%%% END PREAMBLE %%%%%%%%%%%%%%%%%%%%%%%%%
%%%%%%%%%%%%%%%%%%%%%%%%%%%%%%%%%%%%%%%%%%%%%%%%%%
\section{Conclusion}
Adopting JIRA has benefited ME by reducing response time, improving record keeping and sharpening communication. Although JIRA has benefited the organization by improving efficiency, lack of education resulting in underutilisation of dashboards and filters, occasional performance issues causing issue backlogs to pile up, and perceived unreliability for sending emails.  Agile software development methodologies such as Scrum and Kanban are supported by the JIRA suite of applications. JIRA has add-ons that extend functionality will complexity ranging from a single feature to full product.  Interestingly, when JIRA is integrated with Confluence usage of both applications will increase because jumping between application is quick and simple. \\

In order to streamline the adoption of \gls{Confluence}, usage intentions must be conveyed as well as detailed guidance on how to use it. Confluence is wiki software used to create knowledge bases, centralize information into a single system and for technical documentation. Useful features of Confluence include integration with \gls{JIRA}, word processing, and team collaboration.  Connecting JIRA and Confluence applications allows enable users to rapidly switch between applications.  Overall, JIRA and Confluence are software tools that improve productivity and organization within BTI IM.